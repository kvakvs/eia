\chapter*{Вступление}
\markboth{\MakeUppercase{Вступление}}{}
\addcontentsline{toc}{chapter}{Вступление}
\label{chap:introduction}
\pagenumbering{arabic}
\setcounter{page}{1}


\section*{О выполнении приложений}
%\addcontentsline{toc}{section}{On Running Software}
\label{sec:on-running-software}

%There's something rather unique in Erlang in how it approaches failure compared to most other programming languages. There's this common way of thinking where the language, programming environment, and methodology do everything possible to prevent errors. Something going wrong at run-time is something that needs to be prevented, and if it cannot be prevented, then it's out of scope for whatever solution people have been thinking about.
В Erlang есть кое-что уникальное в его подходе к сбоям, если сравнить его с большинством других языков. Существует такой общепринятый ход мыслей, при следованию которому и сам язык, и окружение, в котором работает программист, и методология делают всё возможное, чтобы предотвратить ошибки. Если во время исполнения что-то может пойти не так, то это нужно предотвратить, а если предотвращение невозможно, то оно выходит за пределы любого другого решения, о котором могли бы подумать люди.

%The program is written once, and after that, it's off to production, whatever may happen there. If there are errors, new versions will need to be shipped.
Программа пишется один раз и затем отдаётся на производство (\emph{production}), что бы там с ней ни произошло. Если случатся ошибки, то придётся приготовить и доставить заказчику новые версии.

%Erlang, on the other hand, takes the approach that failures will happen no matter what, whether they're developer-, operator-, or hardware-related. It is rarely practical or even possible to get rid of all errors in a program or a system.\footnote{life-critical systems are usually excluded from this category} If you can deal with some errors rather than preventing them at all cost, then most undefined behaviours of a program can go in that "deal with it" approach.
Erlang, с другой стороны, использует такой подход, при котором сбои считаются неизбежными, независимо от того, что явилось их причиной --- разработчик, оператор или аппаратное обеспечение. Редко считается практичным избавляться от всех ошибок в программе или системе\footnote{Обычно в эту категорию не входят жизненно-важные системы} Если вы можете справиться с ошибками вместо того, чтобы любой ценой их не допустить, то подавляющее количество неожиданных поведений программы может быть разрешено с помощью этого подхода <<справься с ситуацией сам>>.

%This is where the "Let it Crash"\footnote{Erlang people now seem to favour "let it fail", given it makes people far less nervous.} idea comes from: Because you can now deal with failure, and because the cost of weeding out all of the complex bugs from a system before it hits production is often prohibitive, programmers should only deal with the errors they know how to handle, and leave the rest for another process (a supervisor) or the virtual machine to deal with.
Вот откуда появилась известная идея <<Дай ему упасть>> (\emph{Let it crash})\footnote{Люди из мира Erlang чаще предпочитают альтернативное <<дай ему завершиться неудачей>> (\emph{let it fail}), поскольку такое словосочетание меньше нервирует других людей.}: Потому что вы теперь готовы справиться с неудачным завершением работы алгоритма, и поскольку стоимость выведения всех сложных ошибок из системы до момента её сдачи часто является заоблачной, программисты должны помнить и обрабатывать только те ошибки, к которым они готовы, а остальные ситуации позволить решать другому процессу (наблюдателю) или виртуальной машине.

%Given that most bugs are transient\footnote{131 out of 132 bugs are transient bugs (they're non-deterministic and go away when you look at them, and trying again may solve the problem entirely), according to Jim Gray in \href{http://www.hpl.hp.com/techreports/tandem/TR-85.7.html}{Why Do Computers Stop and What Can Be Done About It?}}, simply restarting processes back to a state known to be stable when encountering an error can be a surprisingly good strategy.
Поскольку большая часть ошибок являются временными\footnote{131 из 132 ошибок являются временными (они недетерминированны и исчезают, как только вы на них пристально посмотрите, а повторная попытка сделать то же самое часто завершается успешно), согласно Jim Gray в его статье \href{http://www.hpl.hp.com/techreports/tandem/TR-85.7.html}{Почему компьютеры останавливаются и что с этим делать?} (на английском).}, простой перезапуск процессов в состояние, известное ранее как стабильное, при обнаружении ошибки может оказаться удивительно удачной стратегией.

%Erlang is a programming environment where the approach taken is equivalent to the human body's immune system, whereas most other languages only care about hygiene to make sure no germ enters the body. Both forms appear extremely important to me. Almost every environment offers varying degrees of hygiene. Nearly no other environment offers the immune system where errors at run time can be dealt with and seen as survivable.
Erlang является программным окружением, в котором выбранный подход аналогичен иммун\-ной системе человеческого тела, когда большая часть языков программирования только беспоко\-ятся о гигиене, и блокируют доступ бактерий к телу. Обе формы кажутся мне очень важными. Почти любое окружение предлагает различные степени гигиены. Но почти никакое из окружений не предлагает аналог иммунной системы, в которой ошибки времени выполнения решаются на месте и расцениваются как некритичные, после которых система продолжает работу.

%Because the system doesn't collapse the first time something bad touches it, Erlang/OTP also allows you to be a doctor. You can go in the system, pry it open right there in production, carefully observe everything inside as it runs, and even try to fix it interactively. To continue with the analogy, Erlang allows you to perform extensive tests to diagnose the problem and various degrees of surgery (even very invasive surgery), without the patients needing to sit down or interrupt their daily activities.
Поскольку система не обрушивается при первом прикосновении чего-нибудь плохого, Erlang/OTP также позволяет вам стать доктором. Вы можете зайти в систему, открыть крышку прямо на производственном сервере, осторожно изучить всё внутри, не прерывая работы, и даже попытаться интерактивно починить его. Чтобы продолжить эту цепочку аналогий, Erlang позволояет вам выполнять широкий спектр тестов для диагностики проблем и разные уровни хирургии (даже очень глубокую хирургию), без необходимости для пациента ложиться на стол, садиться или прерывать его каждодневные обычные дела.

%This book intends to be a little guide about how to be the Erlang medic in a time of war. It is first and foremost a collection of tips and tricks to help understand where failures come from, and a dictionary of different code snippets and practices that helped developers debug production systems that were built in Erlang.
Эта книга задумана стать небольшой инструкцией на тему того, как стать Erlang-доктором во время войны. Это в первую очередь и в основном --- коллекция подсказок и секретов, которые помогут понять, откуда приходят сбои, и набор фрагментов кода и методик, которые уже помогли другим разработчикам при отладке их рабочих систем, построенных на Erlang.



\section*{Для кого эта книга?}
%\addcontentsline{toc}{section}{Who is this for?}
\label{sec:who-is-this-for}

%This book is not for beginners. There is a gap left between most tutorials, books, training sessions, and actually being able to operate, diagnose, and debug running systems once they've made it to production. There's a fumbling phase implicit to a programmer's learning of a new language and environment where they just have to figure how to get out of the guidelines and step into the real world, with the community that goes with it.
Эта книга не предназначена для начинающих. Существует некоторое расстояние между многими существующими уроками, книгами, учебными занятиями и умением управлять, диагностировать и отлаживать работающие системы после того, как они попали в производство (\emph{production}). Есть некий этап неуверенного нащупывания решений, присущий изучению программистом нового языка и окружения, когда им приходится просто разобраться, как выбираться из пошаговых инструкций и идти в реальный мир с сообществом других людей, которое к этому миру прилагается.

%This book assumes that the reader is proficient in basic Erlang and the OTP framework. Erlang/OTP features are explained as I see fit — usually when I consider them tricky — and it is expected that a reader who feels confused by usual Erlang/OTP material will have an idea of where to look for explanations if necessary\footnote{I do recommend visiting \href{http://learnyousomeerlang.com}{Learn You Some Erlang} or the regular \href{http://www.erlang.org/erldoc}{Erlang Documentation} if a free resource is required}.
Эта книга предполагает, что читатель хорошо освоил базовый Erlang и систему OTP. Возможности Erlang/OTP описываются здесь по желанию автора --- обычно когда автор считает, что в данной ситуации кроется некая хитрость --- и ожидается, что читатель, которого вдруг собьёт с толку некоторый материал об Erlang/OTP, знает где искать пояснения при необходимости\footnote{Я рекомендую посетить сайт \href{http://learnyousomeerlang.com}{Изучай Erlang во имя добра} (имеется в продаже \href{http://dmkpress.com/catalog/computer/programming/functional/978-5-97060-086-3/}{русский перевод книги} на сайте издательства ДМК Пресс), а также \href{http://www.erlang.org/erldoc}{Стандартную документацию Erlang}.}.

%What is not necessarily assumed is that the reader knows how to debug Erlang software, dive into an existing code base, diagnose issues, or has an idea of the best practices about deploying Erlang in a production environment\footnote{Running Erlang in a screen or tmux session is \emph{not} a deployment strategy.}.
Вот что не требуется явно, так это то, что читатель должен знать об отладке программ на Erlang, уметь копаться в существующем исходном коде, диагностировать проблемы или иметь представление о лучших практиках установки Erlang-программ в производственном окружении\footnote{Заметьте, что запуск Erlang в экране screen или tmux \emph{не} является стратегией установки программы.}.


\section*{Как читать эту книгу}
%\addcontentsline{toc}{section}{How To Read This Book}
\label{sec:how-to-read-this-book}

%This book is divided in two parts. 
Книга разделена на две части.

%Part \ref{part:writing-applications} focuses on how to write applications. It includes how to dive into a code base (Chapter \ref{chap:how-to-dive-into-a-code-base}), general tips on writing open source Erlang software (Chapter \ref{chap:building-open-source-erlang-software}), and how to plan for overload in your system design (Chapter \ref{chap:overload}).
Часть \ref{part:writing-applications} сосредоточена на методике написания приложений. Она включает рекомендации по работе с существующим исходным кодом (Глава \ref{chap:how-to-dive-into-a-code-base}), общие советы по написанию программ с открытым исходным кодом (\emph{open source}) на языке Erlang (Глава \ref{chap:building-open-source-erlang-software}), и как учитывать чрезвычайную нагрузку в будущем при проектировании вашей системы (Глава \ref{chap:overload}).


%Part \ref{part:diagnosing-applictions} focuses on being an Erlang medic and concerns existing, living systems. It contains instructions on how to connect to a running node (Chapter \ref{chap:connecting}), and the basic runtime metrics available (Chapter \ref{chap:runtime-metrics}). It also explains how to perform a system autopsy using a crash dump (Chapter \ref{chap:crash-dumps}), how to identify and fix memory leaks (Chapter \ref{chap:memory-leaks}), and how to find runaway CPU usage (Chapter \ref{chap:cpu-hogs}). The final chapter contains instructions on how to trace Erlang function calls in production using \otpapp{recon}\footnote{\href{http://ferd.github.io/recon/}{http://ferd.github.io/recon/} — a library used to make the text lighter, and with generally production-safe functions.} to understand issues before they bring the system down (Chapter \ref{chap:tracing}).
Часть \ref{part:diagnosing-applictions} сосредоточена на вашей роли в качестве Erlang-доктора и касается существующих, живых систем. Она содержит инструкции о том, как подключиться к работающему узлу (Глава \ref{chap:connecting}), и перечень доступных во время выполнения метрик (Глава \ref{chap:runtime-metrics}). Здесь также объясняется как выполнить посмертное вскрытие погибшей системы с помощью аварийного дампа (\emph{crash dump}) (Глава \ref{chap:crash-dumps}), как определить и исправить утечки памяти (Глава \ref{chap:memory-leaks}), и как найти украденное процессорное время (Глава \ref{chap:cpu-hogs}). Последняя глава содержит инструкции о том, как трассировать вызовы функций Erlang на живой производственной системе с помощью приложения \otpapp{recon}\footnote{\href{http://ferd.github.io/recon/}{http://ferd.github.io/recon/} — библиотека, содержащая функции в целом безопасные для использования на работающих системах и помогающая сделать текст книги короче и понятнее.} чтобы понять причины проблем до того, как они приведут к падению системы (Глава \ref{chap:tracing}).

%Each chapter is followed up by a few optional exercises in the form of questions or hands-on things to try if you feel like making sure you understood everything, or if you want to push things further.
После каждой главы имеются несколько необязательных упражнений в виде вопросов или задач, которые можно попробовать решить для того, чтобы проверить ваше понимание материала, или если вы хотите узнать чуточку больше по каждой данной теме.
%%%
%%%
%%%
