%\chapter{Connecting to Remote Nodes}
\chapter{Подключение к удалённым узлам}
\label{chap:connecting}

%Interacting with a running server program is traditionally done in one of two ways. One is to do it through an interactive shell kept available by using a \app{screen} or \app{tmux} session that runs in the background and letting someone connect to it. The other is to program management functions or comprehensive configuration files that can be dynamically reloaded.
Взаимодействие с работающей серверной программой традиционно делается двумя способами. Один способ --- подключиться к интерактивному интерпретатору, который остаётся доступным с помощью приложений мультиплексирования консолей таких, как \app{screen} или \app{tmux}, и выполняет ваше приложение в фоне, но позволяет подключить к нему экран и вводить команды. Другой способ --- добавить функции управления или подробные файлы конфигурации, которые можно перезагружать на ходу.

%The interactive session approach is usually okay for software that runs in a strict Read-Eval-Print-Loop (REPL). The programmed management and configuration approach requires careful planning in whatever tasks you think you'll need to do, and hopefully getting it right. Pretty much all systems can try that approach, so I'll skip it given I'm somewhat more interested in the cases where stuff is already bad and no function exists for it.
Подход с интерактивной сессией обычно хорошо подходит для программ, которые работают в строгом цикле ввод команды-исполнение-печать (\emph{REPL, Read-Eval-Print-Loop}). С другой стороны запрограммированные команды управления и конфигурирования требуют внимательного планирования тех задач, которые по вашему мнению могут понадобиться, и затем надеяться, что этого будет достаточно. Практически все системы могут попробовать этот подход, так что я просто пропущу его поскольку мы больше заинтересованы в тех случаях, когда всё уже плохо и нет заготовленной функции, чтобы это исправить.

%Erlang uses something closer to an "interactor" than a REPL. Basically, a regular Erlang virtual machine does not need a REPL, and will happily run byte code and stick with that, no shell needed. However, because of how it works with concurrency and multiprocessing, and good support for distribution, it is possible to have in-software REPLs that run as arbitrary Erlang processes.
Erlang использует нечто ближе к прямому общению с машиной, чем цикл ввода-исполнения (\emph{REPL}). Проще говоря, обычная виртуальная машина Erlang вообще не требует наличия никакого цикла ввода-исполнения, она будет с радостью исполнять свой байтовый код и ничего кроме этого делать ей не нужно. Однако поскольку она работает с параллельностью и одновременной обработкой данных, и имеет хорошую поддержку распределённого исполнения, то есть возможность запускать встроенные интерактивные интерпретаторы (\emph{REPL}), которые являются обычными процессами на Erlang.

%This means that, unlike a single screen session with a single shell, it's possible to have as many Erlang shells connected and interacting with one virtual machine as you want at a time\footnote{More details on the mechanisms at \href{http://ferd.ca/repl-a-bit-more-and-less-than-that.html}{http://ferd.ca/repl-a-bit-more-and-less-than-that.html}}.
Это означает, что в отличие от обычной единственной сессии с подключенным терминалом пользователя, вполне возможно иметь столько интерактивных интерпретаторов, работающих с одной виртуальной машиной, сколько вам необходимо\footnote{Подробнее об этих механизмах здесь: \href{http://ferd.ca/repl-a-bit-more-and-less-than-that.html}{http://ferd.ca/repl-a-bit-more-and-less-than-that.html}}.

%Most common usages will depend on a cookie being present on the two nodes you want to connect together\footnote{More details at \href{http://learnyousomeerlang.com/distribunomicon\#cookies}{http://learnyousomeerlang.com/distribunomicon\#cookies} or \href{http://www.erlang.org/doc/reference\_manual/distributed.html\#id83619}{http://www.erlang.org/doc/reference\_manual/distributed.html\#id83619}}, but there are ways to do it that do not include it. Most usages will also require the use of named nodes, and all of them will require \emph{a priori} measures to make sure you can contact the node.
Самые распространённые варианты использования будут зависеть от наличия секретного куки (\emph{cookie}) на обоих узлах, которые вы желаете соединить\footnote{Подробнее о куки в книге: \href{http://learnyousomeerlang.com/distribunomicon\#cookies}{http://learnyousomeerlang.com/distribunomicon\#cookies} или в документации
\href{http://www.erlang.org/doc/reference\_manual/distributed.html\#id83619}{http://www.erlang.org/doc/reference\_manual/distributed.html\#id83619}}, но имеются способы сделать это даже без куки. Большинство вариантов будет требовать от вас использования именованных узлов, и все из них потребуют \emph{заранее} принять меры, чтобы у вас была возможность связаться с узлом. 


%\section{Job Control Mode}
\section{Режим управления задачами}
\newcommand{\Circum}{$ ^{\wedge}$}

%The Job Control Mode (JCL mode) is the menu you get when you press \command{\^{}G} in the Erlang shell. From that menu, there is an option allowing you to connect to a remote shell:
Режим управленя задачами (или сокращённо \emph{JCL}) это небольшое текстовое меню, которое выпадает при нажатии в интерактивном интерпретаторе комбинации клавиш \command{\Circum{}G}. В этом меню есть команда подключения к консоли интерпретатора на удалённом узле:

\begin{VerbatimEshell}
(somenode@ferdmbp.local)1>
User switch command
 --> h
  c [nn]            - connect to job
  i [nn]            - interrupt job
  k [nn]            - kill job
  j                 - list all jobs
  s [shell]         - start local shell
  r [node [shell]]  - start remote shell
  q                 - quit erlang
  ? | h             - this message
 --> r 'server@ferdmbp.local'
 --> c
Eshell Vx.x.x  (abort with ^G)
(server@ferdmbp.local)1>
\end{VerbatimEshell}

%When that happens, the local shell runs all the line editing and job management locally, but the evaluation is actually done remotely. All output coming from said remote evaluation will be forwarded to the local shell.
Когда такое происходит, локальный интерпретатор выполняет работу с клавиатурой, редактирование команды и управление задачами на локальном узле, а выполнение самой команды происходит на удалённом, к которому вы подключились. Весь вывод команды перенаправляется обратно на локальный экран на нашем узле.

%To quit the shell, go back in the JCL mode with \command{\^{}G}. This job management is, as I said, done locally, and it is thus safe to quit with \command{\^{}G q}:
Чтобы завершить работу в консоли, вернитесь обратно в режим JCL нажатием \command{\Circum{}G}. Управление задачами, как я писал выше, выполняется локально и можно безопасно выходить нажатием \command{\Circum{}G q}:

\begin{VerbatimEshell}
(server@ferdmbp.local)1>
User switch command
 --> q
\end{VerbatimEshell}

%You may choose to start the initial shell in hidden mode (with the argument \command{-hidden}) to avoid connecting to an entire cluster automatically.
Вы можете решить запустить первую консоль в скрытом режиме (с параметром \command{-hidden}), чтобы избежать автоматического подключения вашего узла ко всем остальным узлам в кластере.


\section{Remsh}

%There's a mechanism entirely similar to the one available through the JCL mode, although invoked in a different manner. The entire JCL mode sequence can by bypassed by starting the shell as follows for long names:
Существует механизм совершенно подобный тому, что доступен через режим JCL, хотя вызывается он несколько иначе. Все шаги, связанные с режимом JCL, можно пропустить если запустить интерпретатор следующим образом (для длинных имён):

\begin{VerbatimText}
erl -name local@domain.name -remsh remote@domain.name
\end{VerbatimText}

%And as follows for short names:
И таким образом для коротких имён:

\begin{VerbatimText}
erl -sname local@domain -remsh remote@domain
\end{VerbatimText}

%All other Erlang arguments (such as \command{-hidden} and \command{-setcookie \$COOKIE}) are also valid. The underlying mechanisms are the same as when using JCL mode, but the initial shell is started remotely instead of locally (JCL is still local). \command{\^{}G}  remains the safest way to exit the remote shell.
Все другие аргументы Erlang (такие, как \command{-hidden} и \command{-setcookie \$COOKIE}) тоже разрешены. Внутренние механизмы действуют точно так же, как и в режиме JCL, но интерпретатор сразу запускается удалённо вместо локального (JCL работает локально). Клавиша \command{\Circum{}G} остаётся самым безопасным способом, чтобы покинуть удалённую консоль интерпретатора.


%\section{SSH Daemon}
\section{SSH-демон}

%Erlang/OTP comes shipped with an SSH implementation that can both act as a server and a client. Part of it is a demo application providing a remote shell working in Erlang.
Erlang/OTP содержит в стандартной поставке реализацию протокола SSH, которая может действовать как в качестве сервера, так и клиента. Часть его является демонстрационным приложением, которое открывает удалённый терминал, работающий с Erlang.

%To get this to work, you usually need to have your keys to have access to SSH stuff remotely in place already, but for quick test purposes, you can get things working by doing:
Обычно требуется, чтобы ключи находились в соответствующей директории, чтобы получить удалённый доступ к SSH. Но для нашего теста будет вполне достаточно выполнить следующие команды:

\begin{VerbatimEshell}
$ mkdir /tmp/ssh
$ ssh-keygen -t rsa -f /tmp/ssh/ssh_host_rsa_key
$ ssh-keygen -t rsa1 -f /tmp/ssh/ssh_host_key
$ ssh-keygen -t dsa -f /tmp/ssh/ssh_host_dsa_key
$ erl
1> application:ensure_all_started(ssh).
{ok,[crypto,asn1,public_key,ssh]}
2> ssh:daemon(8989, [{system_dir, "/tmp/ssh"},
2>                   {user_dir, "/home/ferd/.ssh"}]).
{ok,<0.52.0>}
\end{VerbatimEshell}

%I've only set a few options here, namely \expression{system\_dir}, which is where the host files are, and \expression{user\_dir}, which contains SSH configuration files. There are plenty of other options available to allow for specific passwords, customize handling of public keys, and so on\footnote{Complete instructions with all options to get this set up are available at \href{http://www.erlang.org/doc/man/ssh.html\#daemon-3}{http://www.erlang.org/doc/man/ssh.html\#daemon-3}.}.
Здесь я задал всего лишь несколько параметров, в частности \expression{system\_dir}, которая указывает, где находятся файлы ключей, и \expression{user\_dir}, указывающий директорию с файлами конфигурации SSH. Есть множество других параметров, позволяющих задавать пароли, настраивать управление публичными ключами и так далее\footnote{Полная инструкция по всем параметрам доступна здесь: \href{http://www.erlang.org/doc/man/ssh.html\#daemon-3}{http://www.erlang.org/doc/man/ssh.html\#daemon-3}.}.

%To connect to the daemon, any SSH client will do:
Для подключения к демону подходит любой SSH-клиент:

\begin{VerbatimEshell}
$ ssh -p 8989 ferd@127.0.0.1
Eshell Vx.x.x  (abort with ^G)
1> 
\end{VerbatimEshell}

%And with this you can interact with an Erlang installation without having it installed on the current machine. Just disconnecting from the SSH session (closing the terminal) will be enough to leave. \emph{Do not run} functions such as \expression{q()} or \expression{init:stop()}, which will terminate the remote host.\footnote{This is true for all methods of interacting with a remote Erlang node.}
Таким образом вы сможете взаимодействовать с установленной Erlang-системой без необходимости устанавливать её на вашей локальной машине. Простое отключение SSH-сессии достаточно для того, чтобы покинуть консоль. \emph{Не выполняйте} функции остановки узла, такие как \expression{q()} или \expression{init:stop()}, которые приведут к остановке удалённого сервера\footnote{Этот совет действителен для всех способов взаимодействия с удалённым узлом Erlang.}.

%If you have trouble connecting, you can add the \command{-oLogLevel=DEBUG} option to \app{ssh} to get debug output.
Если у вас возникли проблемы с подключением, можно добавить к \app{ssh} опцию \command{-oLogLevel=DEBUG} для получения подробного пошагового вывода и помощи в поиске проблемы.


%\section{Named Pipes}
\section{Именованные каналы}

%A little known way to connect with an Erlang node that requires no explicit distribution is through named pipes. This can be done by starting Erlang with \app{run\_erl}, which wraps Erlang in a named pipe\footnote{\command{"erl"} is the command being run. Additional arguments can be added after it. For example \command{"erl +K true"} will turn kernel polling on.}:
Малоизвестный способ подключения к узлу Erlang, который не требует никаких особых параметров для распределённого режима --- с помощью именованных каналов (\emph{named pipes}). Чтобы это сделать, надо запустить Erlang командой \app{run\_erl}, которая обернёт его ввод и вывод в именованный канал\footnote{На самом деле выполняется команда \command{"erl"}. Дополнительные аргументы можно добавить в конце команды. Например, \command{"erl +K true"} включит режим kernel poll.}:

\begin{VerbatimEshell}
$ run_erl /tmp/erl_pipe /tmp/log_dir "erl"
\end{VerbatimEshell}

%The first argument is the name of the file that will act as the named pipe. The second one is where logs will be saved\footnote{Using this method ends up calling fsync for each piece of output, which may give quite a performance hit if a lot of IO is taking place over standard output}.
Первый аргумент -- это имя файла, который будет именованным каналом. Второе --- директория, куда будут писаться файлы журналов\footnote{Использование этого метода будет вызывать fsync для каждого фрагмента вывода, что может дать ощутимый удар по производительности, если ваша программа выводит много на экран (стандартный вывод).}.

%To connect to the node, you use the \app{to\_erl} program:
Для подключения к такому узлу используйте программу \app{to\_erl}:

\begin{VerbatimEshell}
$ to_erl /tmp/erl_pipe
Attaching to /tmp/erl_pipe (^D to exit)

1>
\end{VerbatimEshell}

%And the shell is connected. Closing stdio (with \command{\^{}D}) will disconnect from the shell while leaving it running.
И вот, консоль подключена. Закрытие стандартного ввода-вывода (нажатием клавиши \command{\Circum{}D}) отключится от вашей консоли, оставляя программу запущенной.


\section{Упражнения}

\subsection*{\ReviewTitle{}}

\begin{enumerate}
%	\item What are the 4 ways to connect to a remote node?
	\item Какие есть 4 способа подключения к удалённому узлу?
%	\item Can you connect to a node that wasn't given a name?
	\item Можно ли подключиться к узлу, если ему не было дано имя?
%	\item What's the command to go into the Job Control Mode (JCL)?
	\item Какая команда переходит в режим Job Control Mode (JCL)?
%	\item Which method(s) of connecting to a remote shell should you avoid for a system that outputs a lot of data to standard output?
	\item Какой(какие) методы подключения к удалённой консоли не очень подходят, если система выводит много данных на стандартный вывод?
%	\item What instances of remote connections shouldn't be disconnected using \command{\^{}G}?
	\item Какие из удалённых подключений не следует отключать нажатием клавиш \command{\Circum{}G}?
%	\item What command(s) should never be used to disconnect from a session?
	\item Какие команды нельзя использовать для отключения сессии?
%	\item Can all of the methods mentioned support having multiple users connected onto the same Erlang node without issue?
	\item Все ли перечисленные методы поддерживают множество одновременно подключенных к одной Erlang-системе пользователей без проблем?
\end{enumerate}
%%%
%%%
%%%

%%% Cognitive
%%
%% Knowledge: recall facts, terms, basic concepts
%% 
%% - What are the 4 ways to connect to a remote node?
%% - Can you connect to a node that hasn't received a name?
%% - What's the command to go into the Job Control Mode (JCL)?
%%
%% Comprehension: organizing, comparing, translating, interpreting, giving descriptions, and stating the main ideas
%%
%% - What's the advantage of using the SSH daemon to connect to a node?
%% - Which method(s) require the least preparation in advance to be able to connect to a node?
%%
%% Application: Solve problems in new situations by applying acquired knowledge, facts, techniques and rules in a different way
%%
%% - Which method(s) of connecting to a remote shell should you avoid for an output-heavy application?
%%
%% Analysis: break down info, make inferences, find evidence
%%
%% - What instances of remote connections shouldn't be disconnected using ^G?
%% - What command should never be used to disconnect from a session?
%%
%% Synthesis: Compile information together in a different way by combining elements in a new pattern or proposing alternative solutions
%%
%% - Can all of the methods mentioned support having multiple users connected onto the same Erlang node without issue?
%%
%% Evaluation: Present and defend opinions by making judgments about information
%%
%% - Would you be tempted to run your Erlang software from a screen or a tmux session instead of using the tools presented in this chapter?
%% 

%%%
%%%
%%%
